
\usepackage{graphics}               %figure insertion with \resizebox{\includegraphics{file.eps}
\usepackage{graphicx}               %figure insertion with \includegraphics[width=x.cm]{file.eps}
%\usepackage{subfigure}             %subfigures with \subfloat
\usepackage{caption}
\usepackage{subcaption}
\usepackage[intlimits]{amsmath}     %more maths options, e.g. {split} to align multi-line equations
\usepackage{amssymb}                %more maths symbols e.g. hbar and plot symbols
\usepackage{setspace}               %enables use of \doublespace \onehalfspace \singlespace
\usepackage{fancyhdr}               %customises header and allows chapter, section or page number
\usepackage{minitoc}                %chapter contents at start of chapters
\usepackage{chapterbib}             %bibliography at end of chapters, you will need to run bibtex separately on each chapter file though
\usepackage[british,UKenglish,english]{babel}  % For british english hyphenation patterns
\usepackage{amsfonts}
\usepackage{array}
\usepackage{wrapfig}
\usepackage{ragged2e}
%\usepackage[backend=biber]{biblatex}

\usepackage{soul}
\usepackage{tikz}
\usetikzlibrary{calc}
\usetikzlibrary{decorations.pathmorphing}

\usepackage{xfrac}
\usepackage{bigints}
\usepackage{mathabx}

% *************** Some colour definitions ***************
\usepackage{color}
\definecolor{greenyellow}   {cmyk}{0.15, 0   , 0.69, 0   }
\definecolor{yellow}        {cmyk}{0   , 0   , 1   , 0   }
\definecolor{goldenrod}     {cmyk}{0   , 0.10, 0.84, 0   }
\definecolor{dandelion}     {cmyk}{0   , 0.29, 0.84, 0   }
\definecolor{apricot}       {cmyk}{0   , 0.32, 0.52, 0   }
\definecolor{peach}         {cmyk}{0   , 0.50, 0.70, 0   }
\definecolor{melon}         {cmyk}{0   , 0.46, 0.50, 0   }
\definecolor{yelloworange}  {cmyk}{0   , 0.42, 1   , 0   }
\definecolor{orange}        {cmyk}{0   , 0.61, 0.87, 0   }
\definecolor{burntorange}   {cmyk}{0   , 0.51, 1   , 0   }
\definecolor{bittersweet}   {cmyk}{0   , 0.75, 1   , 0.24}
\definecolor{redorange}     {cmyk}{0   , 0.77, 0.87, 0   }
\definecolor{mahogany}      {cmyk}{0   , 0.85, 0.87, 0.35}
\definecolor{maroon}        {cmyk}{0   , 0.87, 0.68, 0.32}
\definecolor{brickred}      {cmyk}{0   , 0.89, 0.94, 0.28}
\definecolor{red}           {cmyk}{0   , 1   , 1   , 0   }
\definecolor{orangered}     {cmyk}{0   , 1   , 0.50, 0   }
\definecolor{rubinered}     {cmyk}{0   , 1   , 0.13, 0   }
\definecolor{wildstrawberry}{cmyk}{0   , 0.96, 0.39, 0   }
\definecolor{salmon}        {cmyk}{0   , 0.53, 0.38, 0   }
\definecolor{carnationpink} {cmyk}{0   , 0.63, 0   , 0   }
\definecolor{magenta}       {cmyk}{0   , 1   , 0   , 0   }
\definecolor{violetred}     {cmyk}{0   , 0.81, 0   , 0   }
\definecolor{rhodamine}     {cmyk}{0   , 0.82, 0   , 0   }
\definecolor{mulberry}      {cmyk}{0.34, 0.90, 0   , 0.02}
\definecolor{redviolet}     {cmyk}{0.07, 0.90, 0   , 0.34}
\definecolor{fuchsia}       {cmyk}{0.47, 0.91, 0   , 0.08}
\definecolor{lavender}      {cmyk}{0   , 0.48, 0   , 0   }
\definecolor{thistle}       {cmyk}{0.12, 0.59, 0   , 0   }
\definecolor{orchid}        {cmyk}{0.32, 0.64, 0   , 0   }
\definecolor{darkorchid}    {cmyk}{0.40, 0.80, 0.20, 0   }
\definecolor{purple}        {cmyk}{0.45, 0.86, 0   , 0   }
\definecolor{plum}          {cmyk}{0.50, 1   , 0   , 0   }
\definecolor{violet}        {cmyk}{0.79, 0.88, 0   , 0   }
\definecolor{royalpurple}   {cmyk}{0.75, 0.90, 0   , 0   }
\definecolor{blueviolet}    {cmyk}{0.86, 0.91, 0   , 0.04}
\definecolor{periwinkle}    {cmyk}{0.57, 0.55, 0   , 0   }
\definecolor{cadetblue}     {cmyk}{0.62, 0.57, 0.23, 0   }
\definecolor{cornflowerblue}{cmyk}{0.65, 0.13, 0   , 0   }
\definecolor{midnightblue}  {cmyk}{0.98, 0.13, 0   , 0.43}
\definecolor{navyblue}      {cmyk}{0.94, 0.54, 0   , 0   }
\definecolor{royalblue}     {cmyk}{1   , 0.50, 0   , 0   }
\definecolor{blue}          {cmyk}{1   , 1   , 0   , 0   }
\definecolor{cerulean}      {cmyk}{0.94, 0.11, 0   , 0   }
\definecolor{cyan}          {cmyk}{1   , 0   , 0   , 0   }
\definecolor{processblue}   {cmyk}{0.96, 0   , 0   , 0   }
\definecolor{skyblue}       {cmyk}{0.62, 0   , 0.12, 0   }
\definecolor{turquoise}     {cmyk}{0.85, 0   , 0.20, 0   }
\definecolor{tealblue}      {cmyk}{0.86, 0   , 0.34, 0.02}
\definecolor{aquamarine}    {cmyk}{0.82, 0   , 0.30, 0   }
\definecolor{bluegreen}     {cmyk}{0.85, 0   , 0.33, 0   }
\definecolor{emerald}       {cmyk}{1   , 0   , 0.50, 0   }
\definecolor{junglegreen}   {cmyk}{0.99, 0   , 0.52, 0   }
\definecolor{seagreen}      {cmyk}{0.69, 0   , 0.50, 0   }
\definecolor{green}         {cmyk}{1   , 0   , 1   , 0   }
\definecolor{forestgreen}   {cmyk}{0.91, 0   , 0.88, 0.12}
\definecolor{pinegreen}     {cmyk}{0.92, 0   , 0.59, 0.25}
\definecolor{limegreen}     {cmyk}{0.50, 0   , 1   , 0   }
\definecolor{yellowgreen}   {cmyk}{0.44, 0   , 0.74, 0   }
\definecolor{springgreen}   {cmyk}{0.26, 0   , 0.76, 0   }
\definecolor{olivegreen}    {cmyk}{0.64, 0   , 0.95, 0.40}
\definecolor{rawsienna}     {cmyk}{0   , 0.72, 1   , 0.45}
\definecolor{sepia}         {cmyk}{0   , 0.83, 1   , 0.70}
\definecolor{brown}         {cmyk}{0   , 0.81, 1   , 0.60}
\definecolor{tan}           {cmyk}{0.14, 0.42, 0.56, 0   }
\definecolor{gray}          {cmyk}{0   , 0   , 0   , 0.50}
\definecolor{darkgray}          {cmyk}{0   , 0   , 0   , 0.90}
\definecolor{black}         {cmyk}{0   , 0   , 0   , 1   }
\definecolor{white}         {cmyk}{0   , 0   , 0   , 0   }
\definecolor{mygreen}{rgb}{0,0.6,0}
\definecolor{mygray}{rgb}{0.5,0.5,0.5}
\definecolor{mymauve}{rgb}{0.58,0,0.82}

\usepackage[hyphens]{url}
% Use package hyperref
\usepackage[plainpages=false,pdfpagelabels,bookmarksnumbered,hyperindex,
        colorlinks=true, % true for colour
        linkcolor=sepia,
        citecolor=sepia,
        filecolor=maroon,
        %pagecolor=red,
        urlcolor=sepia,
        %pdftex,
        breaklinks=true,
        unicode]{hyperref}
\usepackage{multirow}
        
% For links that should not be broken use: \href{mailto:xport@tex.stackexchange.com}{\nolinkurl{xport@tex.stackexchange.com}}
\pdfimageresolution=600
\usepackage{thumbpdf}
\usepackage{breakurl}

%=======================DEFINE PAGE====================================================================================
\topmargin      -1.8cm              %top of page margin
\headheight     14.5pt              %running head height
\parskip        2mm                 %paragraph spacing
\oddsidemargin  12mm                %left margin of odd page =28mm +oddsidemargin
\evensidemargin 1mm                 %left margin of even page =29mm +evensidemargin
\textheight     250mm               %height of text box on page
\textwidth      145mm               %width of text across the page =textwidth -5mm
\parindent      8mm                 %paragraph indentation width
\pagestyle{fancy}                   %define headers, Left/Right-Odd Page, Left/Right-Even Page
    \fancyhead{}                    %Reset fancy fields
    \fancyhead[LO]{S. Cooper}        %Other position is [RE]
    \fancyhead[RO]{\rightmark}      %rightmark is the section name
    \fancyhead[LE]{\leftmark}       %leftmark is the chapter name
    \renewcommand{\headrulewidth}{0.5mm}
        \headsep=   1.3cm           %Text separation from header
%    \renewcommand{\footrulewidth}{0.5mm}
    \cfoot{\thepage}
        \footskip=  1.3cm           %Text separation from footer

%===Redefine default figure placement================================================================================
% This allows Latex to put the figure where you define it by default i.e. [h] ->here
\makeatletter
\def\fps@figure{htbp} %default is {tbp}
\makeatother

%==========================DEFINE PERSONALISED COMMANDS===============================================================
%Instead of typing out commonly used long commands/items that are long you can define your own commands to use.
%Notes: (1)\ensuremath is used in case you want to use the command while already in a math environment
%e.g. $\theta =30\degr +45\degr$
%(2) Latex wont put a space between this and the next word so I put it in manually and define another command name with
%a s suffix when I specifically don't need a space e.g. at the end of a sentence. This can cause issues though as Latex
% wont break the words joined by the tilde apart for line splitting.
\newcommand{\sq}{\ensuremath{^2~}}
\newcommand{\sqs}{\ensuremath{^2}}
\newcommand{\cb}{\ensuremath{^3~}}
\newcommand{\cbs}{\ensuremath{^3}}
\newcommand{\x}{\ensuremath{\times}}
\newcommand{\degr}{\ensuremath{^\circ}~}
\newcommand{\degrs}{\ensuremath{^\circ}}
\newcommand{\prox}{$\approx$}
\newcommand{\degc}{$^{\circ}$C~}
\newcommand{\degcs}{$^{\circ}$C}

% DeclareMathSizes
\DeclareMathSizes{10}{9}{8}{5}

\usepackage[printonlyused,withpage]{acronym} 	% Acronym package
\usepackage{longtable} 				% Use of long tables
\usepackage{booktabs} 				% midrule toprule bottomrule etc. for tables

% Load natbib
\usepackage[round]{natbib}                   %bibiography package
\bibpunct{[}{]}{,}{n}{}{;}
%\renewcommand\bibname{References}     %Changes the Bibliography name to References instead

% *************** Add reference to page number at which bibliography entry is cited ***************
%\usepackage{citeref}
%\renewcommand{\bibitempages}[1]{\newblock {\scriptsize [\mbox{cited at p.\ }#1]}}

\newcommand{\figurerefname}{figure}
\newcommand{\pagerefname}{page}
\newcommand{\tablerefname}{table}
\newcommand{\fref}[1]{\figurerefname~\ref{#1}}
\newcommand{\pref}[1]{\pagerefname~\pageref{#1}}

% New column type
\newcolumntype{C}[1]{>{\centering}m{#1}}

% *************** Load packages ***************
\usepackage{epsfig}
\usepackage{amsthm}
\usepackage{stmaryrd}
\usepackage[figuresright]{rotating}
\usepackage{multirow}
\usepackage{latexsym}
\usepackage{lscape}
\usepackage[utf8]{inputenc}
\usepackage[T1]{fontenc}
%\usepackage{epstopdf}
\usepackage{setspace}
\usepackage{footnote}

% \usepackage{enumitem}
% \setlist{itemsep=2pt,parsep=1pt,topsep=1pt}
\usepackage{enumerate}
\usepackage{upgreek}

\usepackage{listings} % to display computer code in the text
\lstdefinestyle{full}{ %
  backgroundcolor=\color{white},   % choose the background color; you must add \usepackage{color} or \usepackage{xcolor}
  basicstyle=\footnotesize,        % the size of the fonts that are used for the code
  breakatwhitespace=false,         % sets if automatic breaks should only happen at whitespace
  breaklines=true,                 % sets automatic line breaking
  captionpos=b,                    % sets the caption-position to bottom
  commentstyle=\color{mygreen},    % comment style
  deletekeywords={...},            % if you want to delete keywords from the given language
  escapeinside={\%*}{*)},          % if you want to add LaTeX within your code
  extendedchars=true,              % lets you use non-ASCII characters; for 8-bits encodings only, does not work with UTF-8
  %frame=single,                    % adds a frame around the code
  keepspaces=true,                 % keeps spaces in text, useful for keeping indentation of code (possibly needs columns=flexible)
  keywordstyle=\color{blue},       % keyword style
  language=C,                 % the language of the code
  morekeywords={*,...},            % if you want to add more keywords to the set
  numbers=left,                    % where to put the line-numbers; possible values are (none, left, right)
  numbersep=5pt,                   % how far the line-numbers are from the code
  numberstyle=\tiny\color{mygray}, % the style that is used for the line-numbers
  rulecolor=\color{black},         % if not set, the frame-color may be changed on line-breaks within not-black text (e.g. comments (green here))
  showspaces=false,                % show spaces everywhere adding particular underscores; it overrides 'showstringspaces'
  showstringspaces=false,          % underline spaces within strings only
  showtabs=false,                  % show tabs within strings adding particular underscores
  stepnumber=1,                    % the step between two line-numbers. If it's 1, each line will be numbered
  stringstyle=\color{mymauve},     % string literal style
  tabsize=2,                       % sets default tabsize to 2 spaces
  title=\lstname                   % show the filename of files included with \lstinputlisting; also try caption instead of title
}

\lstdefinestyle{customc}{
  belowcaptionskip=1\baselineskip,
  breaklines=true,
  xleftmargin=\parindent,
  language=C,
  numbers=left,                    % where to put the line-numbers; possible values are (none, left, right)
  numbersep=5pt,                   % how far the line-numbers are from the code
  showstringspaces=false,
  basicstyle=\footnotesize\ttfamily,
  keywordstyle=\bfseries\color{green!40!black},
  commentstyle=\itshape\color{purple!40!black},
  identifierstyle=\color{blue},
  stringstyle=\color{orange},
}

\lstdefinestyle{customasm}{
  belowcaptionskip=1\baselineskip,
  frame=L,
  xleftmargin=\parindent,
  language=[x86masm]Assembler,
  basicstyle=\footnotesize\ttfamily,
  commentstyle=\itshape\color{purple!40!black},
}

\lstdefinestyle{customlatex}{
  belowcaptionskip=1\baselineskip,
  breaklines=true,
  xleftmargin=\parindent,
  language=TeX,
  numbers=none,                    % where to put the line-numbers; possible values are (none, left, right)
  numbersep=5pt,                   % how far the line-numbers are from the code
  showstringspaces=false,
  basicstyle=\footnotesize\ttfamily,
  keywordstyle=\bfseries\color{green!40!black},
  commentstyle=\itshape\color{purple!40!black},
  identifierstyle=\color{blue},
  stringstyle=\color{orange},
}

% Packages required for presentation algorithms in pseudo-code
\usepackage[chapter]{algorithm}
\usepackage{algorithmic}

% Library for drawing diagrams
\usepackage{tikz}
\usetikzlibrary{shapes,arrows}

% Add here new theorems such as theorem, lemma, proposition, corollary
\theoremstyle{plain}
\theoremstyle{definition}

\newtheorem{defn}{Definition}
\newtheorem{exmp}{Example}
\theoremstyle{remark}
\newtheorem{rem}{Remark}
\newtheorem{note}{Note}
\newtheorem{theorem}{Theorem}[section]
\newtheorem{lemma}[theorem]{Lemma}
\newtheorem{proposition}[theorem]{Proposition}
\newtheorem{corollary}[theorem]{Corollary}
\newtheorem{definition}{Definition}


% Placing figures - changes the margins around figures and figure placements
\setcounter{topnumber}{2} 		% default 2
\setcounter{bottomnumber}{2} 		% default 1
\setcounter{totalnumber}{6} 		% default 3
\renewcommand{\topfraction}{0.9} 	% default 0.7
\renewcommand{\bottomfraction}{0.85}	% default?
\renewcommand{\textfraction}{0.15}	% default 0.2
\renewcommand{\floatpagefraction}{0.7}	% default 0.5

\usepackage[section]{placeins}
\usepackage{afterpage}
\usepackage{float}

\makesavenoteenv{tabular}
\usepackage{perpage} 		%the perpage package
\MakePerPage{footnote} 		%the perpage package command